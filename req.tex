\documentclass[12pt,a4paper]{article}
\usepackage[utf8]{inputenc}
\usepackage[russian]{babel}
\usepackage[OT1]{fontenc}
\usepackage{amsmath}
\usepackage{amsfonts}
\usepackage{amssymb}

\begin{document}

\title{Функциональные требования}
\maketitle

\begin{itemize}
\item Система, позволяющая группе разработчиков управлять разработкой программных проектов

\item Система должна предоставлять веб-интерфейс для доступа

\item Поддерживаемая ОС --- Linux

\item Система работает со следующими терминами:
\begin{itemize}
\item Проект.  У каждого проекта есть определенная команда разработчиков, тестировщиков и один менеджер. У проекта определены различные майлстоуны. К каждому проекту могут быть привязаны тикеты и багрепорты.

\item Майлстоун --- очередная итерация в разработке проекта. Привязан к определенным датам. К майлстоунам привязаны определенные тикеты. Майлстоун имеет определенный статус: активен или закрыт. Может быть закрыт только когда все его тикеты выполнены.

\item Тикет --- определенное задание для разработчиков. МОжет быть выдано определенной группе разработчиков. Привязан к определенному проекту и майлстоуну. Иметь статус: новый, принятый, в процессе выполнения, выполнен.

\item Багрепорт --- отчет о найденном баге. Привязан к определенному проекту. Имеет статус: новый, исправленный, протестированный, закрытый.
\end{itemize}

\item Система определяет следующие типы пользователей:
\begin{itemize}
\item Менеджер
\item Разработчик
\item Тестировщик
\end{itemize}

\item Все пользователи системы должны иметь возможность:
\begin{itemize}
\item Зарегистрироваться
\item Просмотреть все проекты в которых они участвуют
\end{itemize}

\item Функции менеджера:
\begin{itemize}
\item Создание проекта
\item Привязка разработчиков и тестировщиков к проекту
\item Управление майлстоунами
\item Создание тикетов для разработчиков
\end{itemize}

\item Функции разработчика:
\begin{itemize}
\item Создание тикетов
\item Выполнение тикетов
\item Создание багрепортов
\item Исправление багрепортов
\end{itemize}

\item Функции тестировщика:
\begin{itemize}
\item Тестирование проекта
\item Создание и проверка багрепортов
\end{itemize}

\end{itemize}

\end{document}