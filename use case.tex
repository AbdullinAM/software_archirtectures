\documentclass[12pt,a4paper]{article}
\usepackage[utf8]{inputenc}
\usepackage[russian]{babel}
\usepackage[OT1]{fontenc}
\usepackage{amsmath}
\usepackage{amsfonts}
\usepackage{amssymb}

\begin{document}

\title{Варианты использования}
\maketitle

Разработка проекта:
\begin{enumerate}
\item Менеджер создает новый проект
\item Менеджер добавляет разработчиков в проект
\item Менеджер добавляет тестировщиков в проект
\item \label{st}Менеджер определяет майлстоуны проекта и назначает даты
\item Менеджер выдает тикеты разработчикам
\item \label{en}Разработчики выполняют тикеты ближайшего майлстоуна
\item Шаги \ref{st} - \ref{en} итеративно повторяются
\end{enumerate}

Выполнение тикета:
\begin{enumerate}
\item Менеджер/разработчик создает новый тикет и определяет разработчика. Тикет получает статус "новый"
\item Разработчик получает уведомление о новом тикете и меняет его статус на "принят"
\item Разработчик приступает к выполнению задания. Тикет получает статус "в процессе выполнения"
\item Разработчик выполняет задание и меняет статус тикета на "выполнен"
\item Менеджер проверяет и подтверждает выполнение задания. Тикет получает задание "закрыт" 
\end{enumerate}

Обработка багрепорта:
\begin{enumerate}
\item Тестировщик создает багрепорт описанием бага. Багрепорт получает статус "новый"
\item Разработчик принимает уведомление и меняет статус багрепорта на "активен"
\item Разработчик исправляет баг и меняет статус на "исправлен"
\item Тестировщик проверяет исправление и закрывает багрепорт указав ему статус "закрыт"
\end{enumerate}

\end{document}